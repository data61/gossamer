\documentclass{bioinfo}
\copyrightyear{2011}
\pubyear{2011}

\usepackage{multirow}
\def\Gossamer{\textit{Gossamer}\ }
\def\rhomer{$\rho$-mer\ }
\def\rhomers{$\rho$-mers\ }
\def\kmer{$k$-mer\ }

\begin{document}
\firstpage{1}

\title[Gossamer]{Gossamer - A Resource Efficient {\em de novo} Assembler}
\author[Sample \textit{et~al}]{Thomas Conway\footnote{to whom correspondence should be addressed}\ , Bryan Beresford-Smith,
        Jeremy Wazny,        Andrew Bromage and        Justin Zobel }
\address{NICTA Victoria Research Laboratory, Department of Computer Science and Software Engineering, The University of Melbourne, Parkville, Australia\\
}


\history{Received on XXXXX; revised on XXXXX; accepted on XXXXX}

\editor{Associate Editor: XXXXXXX}


\maketitle

\begin{abstract}

\section{Motivation:}

\section{Results:}

\section{Availability:}
Gossamer is available for non-commercial use 
from http://www.genomics.csse.unimelb.edu.au/product-gossamer.php.

\section{Contact:} \href{tom.conway@nicta.com.au}{tom.conway@nicta.com.au}
\end{abstract}

\vspace{-1em}
\section{Introduction}

Space is a pressing issue.

Missassemblies are a problem.
There is usually a tradeoff between longer contigs and the likelihood of misassemblies.
It is often the case where a misassembly occur that the misassembled contig is composed of
two or more substantial correct sequences connected by short spurious sequences.
In most cases, in practice

\section{Methods}

\section{Results}

\section*{Acknowledgement}
%Text Text Text Text Text Text  Text Text.  \citealp{Boffelli03} might want to know about  text text text text

%\paragraph{Funding\textcolon} 
National ICT Australia (NICTA) is funded by the Australian Government's Department of Communications; 
Information Technology and the Arts;  
Australian Research Council through Backing Australia's Ability; 
ICT Centre of Excellence programs.

\bibliographystyle{natbib}

%\bibliographystyle{achemnat}
%\bibliographystyle{plainnat}
%\bibliographystyle{abbrv}
%\bibliographystyle{bioinformatics}
%
%\bibliographystyle{plain}
%

\vspace{-1em}
\bibliography{paper}


% Nucleic Acids Res 34:1-9 (2006) Riley M, Abe T, Arnaud MB, Berlyn MK, Blattner FR, et al.
% "Escherichia coli K-12: a cooperatively developed annotation snapshot--2005." 

\end{document}
