\documentclass{article}

\begin{document}

\section*{Review 1}

Reviewer 1 asks for no specific changes.

\section*{Review 2}

Reviewer 2 does not specifically ask for changes, but notes that he/she believes we claim there is no
``intelligent'' competing method, and suggests a more statistically oriented approach would be
potential further work.

In the discussion we have more clearly elucidated the relationship between our method and the Tophat based one.

We agree that a statistically principaled approach would probably improve our existing technique.
Accordingly, we have sketched this in the concluding part of the paper.

\section*{Review 3}

Reviewer 3 says that Section 2.2.1, in which we give a general definition of k-mer canonicalization,
is not clear. We have added some explanatory text before the definition which we believe clarifies the intent of the definition.

Reviewer 3 suggests that the results would be improved with one or
two detailed examples of where the Tophat based analysis fails and
the Xenome approach succeeds.
We have added such an example in the last paragraph of the discussion. 
We have also added some comments to the Results section about the mappings (using Blat) of those reads which were classified as human only by
xenome.

As suggested by Reviewer 3 we have added some comments in the Discussion on how our method relates to approaches for dealing with
multi-mapping reads.

\section*{Review 4}

Reviewer 4 indicates that the relationship between repeat resolution techniques (which are similar
to the multi-mapping read resolution techniques) and our technique should be discussed.
We have added such a discussion and included references to some relevant papers.

\end{document}
