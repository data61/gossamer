\documentclass{article}

\begin{document}

Second Generation Sequencing enables the collection of large quantities of sequence data.
While such data sets are incredibly rich, often researchers are interested in answering specific questions about the data.
Typically, a researcher with 100s of gigabytes of RNA-Seq data may be interested in SNPs and relative expression for only a small number of specific genes.
In such a scenario, it is unnecessarily time consuming to process the entire data set against a whole reference set of genes, only to discard most of the analysis products.

Furthermore, to answer some questions detailed alignments may not be necessary. For instance, to determine relative gene expression, simply extracting and counting the relevant subset of reads may be sufficient.
While other questions may require assembly or alignment to be performed, applying these algorithms to a subset of the total collection of reads can yield substantial computational savings, provided that the subset can be extracted efficiently.

We have developed a tool, \textit{electus}, which allows the user to quickly and sensitively extract a relevant selection of reads. Using our previously published representation for \textit{k}-mer sets \cite{ConwayBromage2011}, our tool uses a \textit{k}-mer decomposition of the reference sequence(s) of interest to yield crude but effective read alignments.

We have tested this approach in the analysis of a number of prostate cancer RNA-Seq data sets,
and show that for analysis targeted to the genes involved in known fusions,
\textit{electus} enables us to efficiently produce files containing just those reads that map to the nominated genes, allowing the fpkm \cite{fpkm} measure of gene expression to be computed quickly.

\end{document}
